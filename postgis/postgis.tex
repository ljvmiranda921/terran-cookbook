%%%%%%%%%%%%%%%%%%%%%%%%%%%%%%%%%%%%%%%%%%%%%%%%%%%%%%%%%%%%%%%%%%%%%%
% PostGIS Cheatsheet 
%
% Source: Lester James V. Miranda (ljvmiranda@gmail.com) 
% 
% A one-size-fits-all PostGIS cheat sheet. Kept to two pages, so it 
% can be printed (double-sided) on one piece of paper
% 
% Feel free to distribute this example, but please keep the referral
% to ljvmiranda@gmail.com 
% 
%%%%%%%%%%%%%%%%%%%%%%%%%%%%%%%%%%%%%%%%%%%%%%%%%%%%%%%%%%%%%%%%%%%%%%

\documentclass[10pt,landscape]{article}
\usepackage{amssymb,amsmath,amsthm,amsfonts}
\usepackage{multicol,multirow}
\usepackage{calc}
\usepackage{listings}
\usepackage{ifthen}
\usepackage[landscape]{geometry}
\usepackage[colorlinks=true,citecolor=blue,linkcolor=blue]{hyperref}
\usepackage{courier}
\lstset{basicstyle=\footnotesize\ttfamily,breaklines=true,language=SQL}

\ifthenelse{\lengthtest { \paperwidth = 11in}}
    { \geometry{top=.5in,left=.5in,right=.5in,bottom=.5in} }
	{\ifthenelse{ \lengthtest{ \paperwidth = 297mm}}
		{\geometry{top=1cm,left=1cm,right=1cm,bottom=1cm} }
		{\geometry{top=1cm,left=1cm,right=1cm,bottom=1cm} }
	}
\pagestyle{empty}
\makeatletter
\renewcommand{\section}{\@startsection{section}{1}{0mm}%
                                {-1ex plus -.5ex minus -.2ex}%
                                {0.5ex plus .2ex}%x
                                {\normalfont\large\bfseries}}
\renewcommand{\subsection}{\@startsection{subsection}{2}{0mm}%
                                {-1explus -.5ex minus -.2ex}%
                                {0.5ex plus .2ex}%
                                {\normalfont\normalsize\bfseries}}
\renewcommand{\subsubsection}{\@startsection{subsubsection}{3}{0mm}%
                                {-1ex plus -.5ex minus -.2ex}%
                                {1ex plus .2ex}%
                                {\normalfont\small\bfseries}}
\makeatother
\setcounter{secnumdepth}{0}
\setlength{\parindent}{0pt}
\setlength{\parskip}{0pt plus 0.5ex}
% -----------------------------------------------------------------------

\title{A PostGIS Cookbook for BigQuery}

\begin{document}

\raggedright
\footnotesize

\begin{center}
     \Large{\textbf{A PostGIS Cookbook for BigQuery}} \\
\end{center}
\begin{multicols}{3}
\setlength{\premulticols}{1pt}
\setlength{\postmulticols}{1pt}
\setlength{\multicolsep}{1pt}
\setlength{\columnsep}{2pt}

\section{Introduction}

This section gives a short description of PostGIS and BigQuery. The two are
definitely a match made in heaven, especially with the latter having
geospatial functions that you can call at scale. Instead of maintaining
a PostGIS server on the cloud, we can just load data on BigQuery, and perform
all our spatial reasoning there: no-ops and no database administration.

\subsection{What is PostGIS?}
PostGIS is an open-source software program that adds support for geographic
objects to the PostgreSQL object-relational database. It allows us to do
spatial queries (and perform spatial reasoning), in a more scalable
manner.

\subsection{What is BigQuery?}
BigQuery is a data warehouse that delivers super-fast results from SQL queries,
which it accomplishes using a powerful engine dubbed Dremel. In BQ, you
don't have to spin up clusters of machines to work with data. 

\section{Cookbook}
This section includes some easy-to-follow recipes to perform some spatial
queries. There may be no logic in their arrangement whatsoever, so use
what you can see. Lastly, some SQL commands here may be based on BigQuery
rather than good old-fashioned PostgreSQL.

\subsection{Transforming \texttt{STRING} into \texttt{GEOGRAPHY}}
Sometimes, lat-lon coordinates are loaded as type \texttt{STRING}
rather than \texttt{GEOGRAPHY}. Here's a command that lets you
convert them into proper data types: 

\begin{lstlisting}
SELECT
    *,
    ST_GeogPoint(
        CAST (longitude AS FLOAT64),
        CAST (latitude AS FLOAT64)
    ) as g
FROM
    table
\end{lstlisting}

\subsection{Sanitizing bad values in raw coordinates}

Sometimes you'll have lat-lon (as \texttt{STRING}s) that are troublesome
when \texttt{CAST}ing. It's better to just replace them

\begin{lstlisting}
SELECT
    REPLACE(latitude, "*", "") AS CLEAN_LAT
    REPLACE(longitude, "*", "") AS CLEAN_LON
    * EXCEPT (LATITUDE, LONGITUDE)
FROM table
\end{lstlisting}

\subsection{Getting the distance of a point to a line} This is useful when
finding the distance between a certain point-of-interest (PoI) to a line
geography like a road.  In this example, we look for the nearest distance
to a \texttt{trunk} road. (\textbf{Estimated time:} around 2 minutes)

\begin{lstlisting}
WITH pairs AS (
  WITH roads AS (
    SELECT osm_id, roads.WKT AS WKT
    FROM `osm.gis_osm_roads_free_1` AS roads
    WHERE roads.fclass = 'trunk')
  SELECT
    ST_DISTANCE(points.WKT, roads.WKT) AS dist,
    points.WKT AS points_WKT,
    points.pkey AS house_pkey,
    roads.WKT AS roads_WKT,
    roads.osm_id AS osm_id
  FROM
    `table` AS points,
    roads AS roads
  WHERE
    ST_DISTANCE(points.WKT, roads.WKT) < 10 * 1000)
SELECT
  ARRAY_AGG(pairs ORDER BY dist LIMIT 1)[OFFSET(0)] event
FROM pairs AS pairs
GROUP BY pairs.house_pkey
\end{lstlisting}

As you can see, we're setting a filter of \texttt{dist < 10 * 1000}.
That means, we're removing distances less than 10km. This is for optimizing
our queries.

\subsection{Getting the number of lines within a particular point radius or
``buffer''} Turns
out, BigQuery doesn't have an \texttt{ST\_BUFFER} command. Here, we present a
workaround. Suppose we want to count the number of \texttt{trunk} roads within
a particular distance (half a kilometer) 

\begin{lstlisting}
WITH pairs AS (
  WITH roads AS (
    SELECT osm_id, roads.WKT AS WKT
    FROM `osm.gis_osm_roads_free_1` AS roads
    WHERE
      roads.fclass = 'primary'
      OR roads.fclass = 'secondary')
  SELECT
    ST_DISTANCE(points.WKT, roads.WKT) AS dist,
    points.WKT AS points_WKT,
    points.pkey AS house_pkey,
    roads.WKT AS roads_WKT,
    roads.osm_id AS osm_id
  FROM
    `table` AS points,
    roads AS roads
  WHERE
    ST_DISTANCE(points.WKT, roads.WKT) < 0.5*1000)
SELECT
  house_pkey,
  ST_ASTEXT(points_WKT) as house_WKT,
  COUNT(DISTINCT(osm_id)) as num_roads
  FROM pairs
  GROUP BY house_pkey, house_WKT
\end{lstlisting}

\subsection{Getting the distance between two points} If you wish to get the
distance of a point to a nearest point-of-interest (PoI), then this SQL query
is for you!

\begin{lstlisting}
WITH pairs AS(
    WITH pois AS (
        SELECT osm_id, poi.WKT AS WKT
        FROM `osm.gis_osm_pois_free_1` AS poi
        WHERE poi.fclass = 'supermarket')
    SELECT
        pois.osm_id as osm_id,
        ST_DISTANCE(points.WKT, pois.WKT) as dist,
        points.WKT AS points_WKT,
        pois.WKT AS poi_WKT,
        points.pkey as house_pkey
    FROM
        `table` AS points,
        pois AS pois
    WHERE
        ST_DISTANCE(points.WKT, pois.WKT) < 10 * 1000)
SELECT
  ARRAY_AGG(t ORDER BY dist LIMIT 1)[OFFSET(0)] poi
FROM pairs as t
GROUP BY t.house_pkey
\end{lstlisting}

\subsection{Getting the intersection of line entities} This query may
be useful when you want to find the intersection of different roads. The output
of this query is a set of points that you can use for more advanced reasoning.

\begin{lstlisting}
SELECT
  a.osm_id as b_osm_id, 
  b.osm_id  as a_osm_id 
FROM
  `osm.gis_osm_roads_free_1` as a,
  `osm.gis_osm_roads_free_1` as b
WHERE
  ST_INTERSECTS(a.WKT, b.WKT) IS NOT NULL AND
  a.osm_id < b.osm_id
\end{lstlisting}

\section{Best Practices}
Below are some nice hints and practices for optimizing your queries and running
PostGIS-BigQuery smoothly:

\begin{itemize}
    \item Remove unnecessary geographies from your results.
    \item Filter first, before doing a cross join.
    \item Ensure that all operations happen in the primary keys \texttt{pkey}
        rather than the geography.
    \item Create an envelope or bounding box when doing \texttt{within} or
        \texttt{intersect}s operations\textemdash the geographies in our
        Philippine administrative region datasets are not that simple. 
    \item QGIS can import CSVs with WKT strings (e.g. the results of exporting
        a BQ table to a CSV in Google Cloud Storage)
    \item Create a \texttt{wkt\_string} column in your query results if you
        will need to export to QGIS. There appears to be a bug when exporting
        geography columns that will produce something like a \texttt{WKT}
        column, but some of the records will have malformed entries for the
        \texttt{WKT}. 
\end{itemize}


%====== REFERENCES ===============
\nocite{*}
\bibliographystyle{plain}
\bibliography{bibliography}
    
\vfill
\hrule
~\\
Lj V. Miranda, Thinking Machines Data Science\\
Contact: \href{mailto:lj@thinkingmachin.es}{lj@thinkingmachin.es}
\end{multicols}

\end{document}
