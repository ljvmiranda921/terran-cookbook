%%%%%%%%%%%%%%%%%%%%%%%%%%%%%%%%%%%%%%%%%%%%%%%%%%%%%%%%%%%%%%%%%%%%%%
% GDAL/OGR Cheatsheet 
%
% Source: Lester James V. Miranda (ljvmiranda@gmail.com) 
% 
% A one-size-fits-all PostGIS cheat sheet. Kept to two pages, so it 
% can be printed (double-sided) on one piece of paper
% 
% Feel free to distribute this example, but please keep the referral
% to ljvmiranda@gmail.com 
% 
%%%%%%%%%%%%%%%%%%%%%%%%%%%%%%%%%%%%%%%%%%%%%%%%%%%%%%%%%%%%%%%%%%%%%%

\documentclass[10pt,landscape]{article}
\usepackage{amssymb,amsmath,amsthm,amsfonts}
\usepackage{multicol,multirow}
\usepackage{calc}
\usepackage{listings}
\usepackage{ifthen}
\usepackage[landscape]{geometry}
\usepackage[colorlinks=true,citecolor=blue,linkcolor=blue]{hyperref}
\usepackage{courier}
\lstset{basicstyle=\footnotesize\ttfamily,
        breaklines=true, 
        language=bash,
        showstringspaces=false,
        morekeywords={
            apt,
            get,
            gdalwarp,
            sudo,
            apt-get,
            pip,
            ogr2ogr,
            gdal_edit,
            gdal_translate,
            gdal_contrast_stretch,
            gdal_translate,
            conda
        }}
\renewcommand{\familydefault}{\sfdefault}

\ifthenelse{\lengthtest { \paperwidth = 11in}}
    { \geometry{top=.5in,left=.5in,right=.5in,bottom=.5in} }
	{\ifthenelse{ \lengthtest{ \paperwidth = 297mm}}
		{\geometry{top=1cm,left=1cm,right=1cm,bottom=1cm} }
		{\geometry{top=1cm,left=1cm,right=1cm,bottom=1cm} }
	}
\pagestyle{empty}
\makeatletter
\renewcommand{\section}{\@startsection{section}{1}{0mm}%
                                {-1ex plus -.5ex minus -.2ex}%
                                {0.5ex plus .2ex}%x
                                {\normalfont\large\bfseries}}
\renewcommand{\subsection}{\@startsection{subsection}{2}{0mm}%
                                {-1explus -.5ex minus -.2ex}%
                                {0.5ex plus .2ex}%
                                {\normalfont\normalsize\bfseries}}
\renewcommand{\subsubsection}{\@startsection{subsubsection}{3}{0mm}%
                                {-1ex plus -.5ex minus -.2ex}%
                                {1ex plus .2ex}%
                                {\normalfont\small\bfseries}}
\makeatother
\setcounter{secnumdepth}{0}
\setlength{\parindent}{0pt}
\setlength{\parskip}{0pt plus 0.5ex}
% -----------------------------------------------------------------------

\title{GDAL/OGR Cookbook for Scalable GIS}

\begin{document}

\raggedright
\footnotesize

\begin{center}
     \Large{\textbf{GDAL/OGR Cookbook for Scalable GIS}} \\
\end{center}
\begin{multicols}{3}
\setlength{\premulticols}{1pt}
\setlength{\postmulticols}{1pt}
\setlength{\multicolsep}{1pt}
\setlength{\columnsep}{2pt}

\section{Introduction}

This cookbook is intended for geospatial analysts who would like to perform
raster and vector manipulation at scale. Usually, we do things via the
    QGIS desktop in our local machines, but this may not hold given huge
    amounts of data. There will be cases where we need to employ scripting
    techniques in our geospatial workflows, and this is where GDAL/OGR enters
    the picture.  

\tableofcontents

\subsection{What is GDAL?}

\par The Geospatial Data Abstraction Library (GDAL) is a suite of tools that
    allows us to read, write, and manipulate raster and vector data formats.
    Some QGIS functionalities were built on top of it, however, if we want
    scalable (and more automated) workflows for processing geospatial data,
    then we need to use this library. GDAL consists of
    \href{https://www.gdal.org/gdal_utilities.html}{utility
    programs} that you can invoke within the
    command-line. 

\subsection{What is OGR?}

\par OGR Simple Features Library allows us to perform multiple operations in a
variety of vector formats such as ESRI Shapefiles, S-57, SDTS, PostGIS,
    Oracle Spatial, and Mapinfo mid/mif and TAB formats.

\section{Setup \& Installation}

There are a lot of ways to install GDAL in your system, and because of that,
    can lead to confusing installation instructions. Here's our
    recommended way of installing them.

\subsection{Install via Anaconda}

    The painless way of installing GDAL is through the 
    \href{https://conda.io/en/latest/}{\texttt{conda} package manager}.
    Simply execute the following command:

\begin{lstlisting}
conda install gdal
\end{lstlisting}

This should work most of the time. You can check if your installation
    succeeded by running:

\begin{lstlisting}
gdal-config --version
\end{lstlisting}

and check if it installed inside your Anaconda path:

\begin{lstlisting}
gdal-config --prefix
\end{lstlisting}

\subsection{Install GDAL manually}

If you don't rely on the Anaconda distribution nor the \texttt{conda}
package manager, then you should install GDAL manually. First, install some
system-wide dependencies:

\begin{lstlisting}
sudo apt-get install libgdal-dev gdal-bin=2.3.* 
\end{lstlisting}

I suggest looking at the dependency trees for
\href{https://packages.ubuntu.com/trusty/libgdal-dev}{\texttt{libgdal-dev}}
and \href{https://packages.ubuntu.com/trusty/gdal-bin}{\texttt{gdal-bin}}
to better understand how they all fit together. Afterwards, you need to export the following environmental variables

\begin{lstlisting}
export GDAL_DATA=/usr/share/gdal/2.1/
export CPLUS_INCLUDE_PATH=/usr/include/gdal
export C_INCLUDE_PATH=/usr/include/gdal
\end{lstlisting}

Then you can install the Python bindings for GDAL:

\begin{lstlisting}
sudo apt-get install python-gdal
\end{lstlisting}

(optional) You can also install the Python library from \texttt{pip}. This
assumes that all of the above steps were successful:

\begin{lstlisting}
pip install GDAL
\end{lstlisting}

\section{Cookbook}

This section includes some easy-to-follow recipes to perform some geospatial
operations to vector or raster data. There may be no logic in their
arrangement whatsoever, so use what you can see. 

\subsection{Loading data into PostgreSQL}

If you want to store your data in a geodata warehouse, then follow this
command:

\begin{lstlisting}
ogr2ogr -f "PostgreSQL" 
        PG:"dbname='my_database' 
        host='XX.XXX.XXX.XXX' port='5432' 
        user='postgres' 
        password='<insert password>'" 
        file
\end{lstlisting}

There are restrictions in naming the uploaded files. For example, you cannot
precede with a numeric value. So if you upload a file named
\texttt{001.geojson}, it will automatically be renamed into
\texttt{\_001.geojson}.

In order to override that, you can pass the argument \texttt{-nln} and
provide the new name as a parameter.

\subsection{Converting 16-bit rasters into 8-bit}

There are some cases where we'll need to work with a lower-order image
rather than a high-order one. We can accomplish these steps by executing
the following commands:

\begin{lstlisting}
gdalwarp -srcnodata 0 
         -dstalpha 
         file.TIF file_temp.TIF
gdal_edit.py -unsetnodata file_temp.TIF
gdal_translate -scale 0 65535 0 65535 
               -b 3 -b 2 -b 1 -b 5 
               -co PHOTOMETRIC=RGB
               file_temp.TIF file_temp2.TIF
gdal_contrast_stretch -ndv 0 
                    -percentile-range 0.03 0.97 
                    file_temp2.TIF file_temp3.TIF
gdal_translate -b 1 -scale 0 255 0 255 
               -b 2 -scale 0 255 0 255 
               -b 3 -scale 0 255 0 255 
               -b 4 -scale 0 1 0 255 
               file_temp3.TIF new_file.TIF
\end{lstlisting}

From here, you can remove all intermediary files:

\begin{lstlisting}
rm -rf *_temp*.TIF
\end{lstlisting}

%====== REFERENCES ===============
\nocite{*}
\bibliographystyle{plain}
\bibliography{bibliography}
    
\vfill
\hrule
~\\
Lj V. Miranda, Thinking Machines Data Science\\
Contact: \href{mailto:lj@thinkingmachin.es}{lj@thinkingmachin.es}
\end{multicols}

\end{document}
